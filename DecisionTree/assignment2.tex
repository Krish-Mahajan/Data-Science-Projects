\documentclass[]{article}
\usepackage{lmodern}
\usepackage{amssymb,amsmath}
\usepackage{ifxetex,ifluatex}
\usepackage{fixltx2e} % provides \textsubscript
\ifnum 0\ifxetex 1\fi\ifluatex 1\fi=0 % if pdftex
  \usepackage[T1]{fontenc}
  \usepackage[utf8]{inputenc}
\else % if luatex or xelatex
  \ifxetex
    \usepackage{mathspec}
    \usepackage{xltxtra,xunicode}
  \else
    \usepackage{fontspec}
  \fi
  \defaultfontfeatures{Mapping=tex-text,Scale=MatchLowercase}
  \newcommand{\euro}{€}
\fi
% use upquote if available, for straight quotes in verbatim environments
\IfFileExists{upquote.sty}{\usepackage{upquote}}{}
% use microtype if available
\IfFileExists{microtype.sty}{%
\usepackage{microtype}
\UseMicrotypeSet[protrusion]{basicmath} % disable protrusion for tt fonts
}{}
\usepackage[margin=1in]{geometry}
\ifxetex
  \usepackage[setpagesize=false, % page size defined by xetex
              unicode=false, % unicode breaks when used with xetex
              xetex]{hyperref}
\else
  \usepackage[unicode=true]{hyperref}
\fi
\hypersetup{breaklinks=true,
            bookmarks=true,
            pdfauthor={},
            pdftitle={Habits},
            colorlinks=true,
            citecolor=blue,
            urlcolor=blue,
            linkcolor=magenta,
            pdfborder={0 0 0}}
\urlstyle{same}  % don't use monospace font for urls
\setlength{\parindent}{0pt}
\setlength{\parskip}{6pt plus 2pt minus 1pt}
\setlength{\emergencystretch}{3em}  % prevent overfull lines
\setcounter{secnumdepth}{0}

%%% Use protect on footnotes to avoid problems with footnotes in titles
\let\rmarkdownfootnote\footnote%
\def\footnote{\protect\rmarkdownfootnote}

%%% Change title format to be more compact
\usepackage{titling}

% Create subtitle command for use in maketitle
\newcommand{\subtitle}[1]{
  \posttitle{
    \begin{center}\large#1\end{center}
    }
}

\setlength{\droptitle}{-2em}
  \title{Habits}
  \pretitle{\vspace{\droptitle}\centering\huge}
  \posttitle{\par}
  \author{}
  \preauthor{}\postauthor{}
  \date{}
  \predate{}\postdate{}



\begin{document}

\maketitle


\[  
\begin{aligned}  
\textbf{Data Mining:Assignment 2}  
\end{aligned}
\]

\[  
\begin{aligned}
========================
\end{aligned}
\]

\[  
\begin{aligned} 
\textbf{Krishna Mahajan,0003572903}   
\end{aligned}
\]

\section{Q4}\label{q4}

Implementing classification trees and evaluating their accuracy

\textbf{(a)} \textbf{Implement the greedy algorithm that learns a
classification tree given a data set. Assume that all features are
numerical and properly find the best threshold for each split. Use Gini
and information gain, as specified by user, to decide on the best
attribute to split in every step. Stop growing the tree when all
examples in a node belong to the same class or the remaining examples
contain identical features}

\textbf{(soln)}\\Here are following explanation of python Modules that i
coded while implementing the decision tree in sequence.

\paragraph{1) Reading raw data:}\label{reading-raw-data}

-\textgreater{}Functionality of this module is to open the raw data file
containing all the observations with final target variable(Class
Variable).The module read all the data and do the necessarily clean
(such as converting numerical attributed which are loaded strings to
convert back to numerical type).\\-\textgreater{}Module also add header
to the raw data which was loaded if colname is passed as an
input.\\-\textgreater{}Finally after all the cleaning ,Module convert
the loaded the dataset into python list of list format ex\\\[ 
\begin{aligned} 
 \textbf{[      
 ['d11','d12','d13','c1'],\newline   
 ['d21','d21','d23','c2'],\newline    
 [......................],\newline    
 [......................],\newline    
 ]}   
\end{aligned}
\] and dump this list so that that be later used while building the
decision tree

\[
\begin{document}
Something in this document. This paragraph contains no information 
and its purposes is to provide an example on how to insert white 
spaces and lines breaks.\\
When a line break is inserted, the text is not indented, there 
are a couple of extra commands do line breaks. \newline
This paragraph provides no information whatsoever. We are exploring 
line breaks. \hfill \break
And combining two commands
\end{document}
\]

\end{document}
